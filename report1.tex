\documentclass[11pt,a4paper,oneside]{report}


\usepackage{amsmath,amssymb,calc,ifthen}

\usepackage{float}

\usepackage[table,usenames,dvipsnames]{xcolor} % for coloured cells in tables

\usepackage{tikz}

% Allows us to click on links and references!

\usepackage{hyperref}
\hypersetup{
    colorlinks,
    citecolor=black,
    filecolor=black,
    linkcolor=black,
    urlcolor=black
}

% Nice package for plotting graphs
% See excellent guide:
% http://www.tug.org/TUGboat/tb31-1/tb97wright-pgfplots.pdf
\usetikzlibrary{plotmarks}
\usepackage{amsmath,graphicx}
\usepackage{epstopdf}
\usepackage{caption}
\usepackage{subcaption}

% highlight - useful for TODOs and similar
\usepackage{color}
\newcommand{\hilight}[1]{}%\colorbox{yellow}{#1}}


% margin size
\usepackage[margin=1in]{geometry}

\tikzstyle{state}=[circle,thick,draw=black, align=center, minimum size=2.1cm,
inner sep=0]
\tikzstyle{vertex}=[circle,thick,draw=black]
\tikzstyle{terminal}=[rectangle,thick,draw=black]
\tikzstyle{edge} = [draw,thick]
\tikzstyle{lo} = [edge,dotted]
\tikzstyle{hi} = [edge]
\tikzstyle{trans} = [edge,->]

\title{Graphical Models Coursework 1}
\author{
    Razvan Valentin Marinescu\\
    \texttt{razvan.marinescu.14@ucl.ac.uk}
    \and
    David Owen\\
    \texttt{email.address@ucl.ac.uk}
    \and
    Kin Quan\\
    \texttt{email.address@ucl.ac.uk}
}

\begin{document}
\belowdisplayskip=12pt plus 3pt minus 9pt
\belowdisplayshortskip=7pt plus 3pt minus 4pt

\maketitle{}


\section*{Problem 2.5}
Hello

\section*{Problem 2.6}


\section*{Problem 2.7}


\section*{Problem 2.9}
Let $N=3n$ where $n$ is an integer, let us prove the hypothesis by induction on $n$. Consider $n=1$ i.e. when $N=3$. This is trivial as there is three nodes with no edges therefore there are three cliques that are form. Let us now assume that the hypothesis is true for for $N$ therefore there exist a graph $G$ such that it displays all the properties stated in the question. To prove the hypothesis let us  

\section*{Problem 3.3}


\section*{Problem 3.4}


\section*{Problem 3.8}


\section*{Problem 3.9}


\section*{Problem 3.11}


\section*{Problem 3.12}


\section*{Problem 3.13}


\section*{Problem 3.14}


\section*{Problem 3.15}


\section*{Problem 3.17}


\section*{Problem 3.20}


\section*{Problem 3.21}


\section*{Problem 3.22}


\section*{Extra Problem A}


\section*{Extra Problem B}


\end{document}
